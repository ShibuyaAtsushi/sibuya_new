\documentclass[a4paper,11pt]{jsarticle}


% 数式
\usepackage{amsmath,amsfonts}
\usepackage{bm}
% 画像
\usepackage[dvipdfmx]{graphicx}


\begin{document}

\title{工学のための生理学 最終レポート}
\author{1M1-34 6301974\\渋谷享史}
\date{\today}
\maketitle


本レポートでは,「工学のための生理学」レポート作成・提出要領に示された2つのうち,
\begin{center}
  「1」生体機能から未知の工学的発想を抽出することによる、\\
  生体から離れた意外性のある工学システムの創成
  \end{center}
  

について述べる.
\newpage
私は,授業での「速い知覚 遅い知覚」に関する講義を聴講し,非常に興味を持った.
授業内での例は最初は勘違いしていることに気づかないほど,速い知覚によって勘違いが生まれ,驚いた.
生物は,反射的に知覚し即座に反応・行動する機能と,ゆっくりと知覚して認知,分析す
る機能が備わっている.
この機能によって,最初は意志によらず反射的に逃げるなど行動
選択を行うため生き延びやすく,そしてその後
ある程度様々な世界の理を理解してきて,じっくりと分析すること
で最適解を求めるということが可能となっている.
\par 私はこの生体機能から,このようなある意味「学習手法」とも取ることができるととらえ,
機械学習の初期値に関する問題に適応できるのではないかと考えた.\par
機械学習のうちの1つである強化学習と呼ばれる手法は,人間が予め定めた
報酬を与えるルールに対して,確率的に最も多くの報酬を受け取ることができる行動方法,つまり
最適な行動方法を獲得することができるというものである.
しかし,学習する際に多くの試行錯誤を必要とするという特徴が
あるため,実ロボットを用いた学習は行うことが難しかった.
そのための工夫として物理シミュレーションソフトを用いることで学習を行っていたわけであるが,
この学習手法は行動自体を機械が自ら学習するため,人間が
思いもしない動作・身体の使い方を学習することがあり,
生物に似ている動作を獲得するなどから非常に興味深いものであった.そしてその結果か
ら,あえて生物に近い動作を行わせるよう促しながら学習させることで
より効率の良い学習結果が得られることもあった.
\par
そこで私はこの講義を聞いて,強化学習をただ行わせるのではなく,生物で言う脊髄反射的な,
速い知覚の行動とその条件を予め設定しておいた上で強化学習を行うことで,
より生物に近い学習結果が得られるのではないかと考えた.
強化学習によって獲得される行動は経験による自己判断に該当し,
人間がロボットに予め設定した,ある条件になると反射的に行動が行われる部分は,
脊髄反射に該当すると考えると,より生物に近い設定で学習を行うことが可能であると考える.
また,深層学習と強化学習を組み合わせた深層強化学習は,脳のニューロンを模した
アーキテクチャが使われているため,それも用いればますます生物の設定に近い条件となる.
\par
人間による行動設定を加えず学習を行っていることが多いため,この
生体機能から未知の「速い知覚」に関する部分を取り入れるという工学的発想を抽出することによる,
生体とは全く異なる身体を持ったロボットにおいても,動作は生体か
ら着想を得たような工学システムが得られるのではないかと感じた.
近々,脊髄反射に該当する条件・行動を定義し,強化学習実験を行ってみようと考えている.





\end{document}