\documentclass[a4paper,11pt]{jsarticle}


% 数式
\usepackage{amsmath,amsfonts}
\usepackage{bm}
% 画像
\usepackage[dvipdfmx]{graphicx}
% 図の先頭につく,fig.1とか図1とか(キャプション)のカスタマイズ
\usepackage{caption}
\captionsetup[figure]{labelformat=simple, labelsep=space, textfont=normalfont, labelfont=bf}
\renewcommand{\figurename}{Fig.}


\begin{document}

\title{中間報告概要}
\author{Shibuya Atsushi 渋谷享史}
\date{\today}
\maketitle

\section{伝える最終目的}
強化学習を用いて実ロボットを動作させること\par
そのために起こる問題を解決するにはどうしたら良いかを分析しつつ,実ロボットへの転用を目指すこと
\section{話の流れ}
\subsection{導入の部分}
・強化学習は経験を沢山積まなければならないため,実ロボットで経験をつもうとすると,時間もかかるし故障するなども考えられ,コストがかかる.\par
・そこで,シミュレーションを用いるのが一般的である.\par
・しかしシミュレーションで経験を積んでも,実世界との差はあるためそれによって,実際のロボットがシミュレーションで行っていた通りの動きをすることはあまりない(証拠を入れたい).\par
・モデル化が難しい不確かな部分をあえてランダムにして学習させることで現実世界との経験の差異を減らすなど様々な工夫を行うことで実機に転用する手法が提案されてきた.\par
 \par
・シミュレーションソフトは様々あり,最もよく使用されているのは「MuJoCo」(証拠)だが,このシミュレーションを使用しても,(このような)問題があった.\par
・つまり,シミュレーションの制度によって,強化学習の結果が当然変わってきてしまうため,適切なシミュレーション環境下で学習させなければ,実機に転用することは難しい.\par
\subsection{研究のために必要な,取り組むべき内容について}
・この問題を解決したい.実物を動かすことを考えた上で,どの様にシミュレーションを行うことが良いのか.それを探るために,まずは2輪移動ロボットを題材にその動作を学習させ,実機を動かすことを目標として検証する.\par
・そのためには,(比較のために)そのロボットが従来どの様に制御されていたかどうかと,そもそもの強化学習の理論をしっかりと理解しておく必要がある.そのため現在はそれらの勉強をしつつ,今後どの様に進めていくべきかを考えているところである.\par
・この2輪移動ロボットは,従来の制御手法は,4つのセンサの様子を見ながら状況を判断し,他のセンサなどで移動距離や自己位置の推定などを行って制御されている.\par
\subsection{現状の進捗と今後の取組予定}
・現在はその制御を勉強していて,段々と形になってきている段階である.同時に強化学習の勉強も行っているところである.取り組む中で,シミュレーションが正しいか不安な動作が見られる事があった.\par
・制御を理解し,実機も製作してその実機に強化学習での結果を転用させることを目指す中で,どの様に工夫すればよいかを考え,検証していきながら,実機転用を想定したシミュレーションの方法について研究する. \par



\end{document}