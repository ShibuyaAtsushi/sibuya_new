\documentclass[a4paper,11pt]{jsarticle}


% 数式
\usepackage{amsmath,amsfonts}
\usepackage{bm}
% 画像
\usepackage[dvipdfmx]{graphicx}
% 図の先頭につく,fig.1とか図1とか(キャプション)のカスタマイズ
\usepackage{caption}
\captionsetup[figure]{labelformat=simple, labelsep=space, textfont=normalfont, labelfont=bf}
\renewcommand{\figurename}{Fig.}


\begin{document}

\title{強化学習による差動二輪移動ロボットの自動軌道調整制御(またはその中での強化学習を用いた少ないセンサでの同等の制御の実現)}
\author{Shibuya Atsushi 渋谷享史}
\date{\today}
\maketitle

(研究にて扱う対象を狭めて,確実に新規性を出すことを考えた.\\
参考:"https://www.graco.c.u-tokyo.ac.jp/labs/morihata/research\_memo.htm")
2輪移動ロボットが強化学習で少ないセンサから目的の調整動作を学習した.という結果が理想(例:ジャイロ無しで90度旋回(ロボメックのテーマと類似)
\\強化学習で,今できていたこと以上の制御を獲得or強化学習を使って限られたセンサのみで同等の制御を獲得)

\section{問題意識}
\subsection{問題}
目標軌道を走行する差動二輪移動ロボットが何らかの理由で目標軌道がらずれてしまった場合,そのずれを修正することが難しい問題
\subsection{目的}
使用できるセンサの値を使って,強化学習をすることで実現の難しい制御を獲得する.それによってより良い制御を実現することができれば,現状のセンサ構成での行えることが増える.または,センサを積む数を減らすことができ,,コストも削減できる
\subsection{問題の重要性}
従来の制御手法では,制御の目的に合わせて,それに必要なセンサをそれぞれロボットに搭載して制御していた.よって行いたい制御に対して必要不可欠なセンサがあった.本研究はそのようなセンサがなくとも,他のセンサの値を使って同等の制御を得ることで,コストの削減など様々なメリットを享受できる(コストの削減,機能の追加等).
\section{その研究は何に貢献するのか}
(この研究の何が新しいのか,優れているのか,どんな貢献が人類にできるのか,)\\
差動二輪移動ロボットが自動で軌道を修正する新しい手法の提案.
ロボット制御の可能性・自由度を高めることに貢献できる.とあるセンサがなくても,他のセンサの値の組み合わせ等を使って同等の制御を得ることができるなど,

\section{分析のフレームワーク}
\subsubsection{問題を解決するために必要な学問}
線形代数,微分積分,確率統計,機械学習,強化学習,制御工学,電気・電子工学
\subsubsection{先行研究で用いられている手法}

\subsubsection{何をどうやって証明するか}
強化学習を用いて自動軌道修正ができたら,少し大げさな,ズレが生じやすい環境での実験を行い軌道修正できていることを示し,軌道修正機能が実現できたことを証明する

\subsubsection{何を作るのか}
2輪移動ロボット実機を作る.シミュレーション結果と実機の動きを比較する.
2Dシミュレーションで強化学習を行う

\subsubsection{どのようなスケジュールで進めるか}
\begin{itemize}
  \item 2024年
  \begin{itemize}
    \item 4月:研究計画の立案 マウスの突貫製作,足立法 
    \item 5月:学習結果を実機で動作させるプログラムを考える,2Dシミュレータの製作
    \item 6月:ロボメック発表,修論のテーマを決める,2Dシミュレータの製作
    \item 7月:実験
    \item 8月:実験
    \item 9月:実験・SIに向け準備
    \item 10月:修論を書き始める,SI準備
    \item 11月:修士論文・SI
    \item 12月:修士論文・SI
  \end{itemize}
  \item 2025年
  \begin{itemize}
    \item 1月:修論
    \item 2月:修士論文
  \end{itemize}
\end{itemize}

\section{参考文献}
・https://ar5iv.labs.arxiv.org/html/2102.02915
\\
 ・https://www.graco.c.u-tokyo.ac.jp/labs/morihata/research\_memo.htm

\section{現状(現在考えていること)}
強化学習の利点は,得られるセンサの値などを用いて,経験から良さそうな行動をすることができるところにある.\par
そのためそれを活かす案として,少ないセンサの値から,目標軌道に戻るような行動を学習させられないかを考えた.しかし,必要不可欠なものを除いてしまっては,強化学習を使ってもそもそも不可能となってしまう.\par
つまり,これをやるのであれば(強化学習を用いて人ができなかったものを実現するのであれば),情報は足りているはずだが,これらの情報をどう処理して意味を見出して行けばよいかの設計が難しい,という場合に有効であると考えられる.\par
そういったロボットの制御を強化学習に行わせるのが良さそうである.
マイクロマウスでそれをやる場合,どうしたらいいのだろうか.マイクロマウス競技のことは忘れて,2輪移動ロボットとして考えるので,挙げてみると,\par
倉庫の移動をするロボット,飲食店の配膳ロボットなどが思い浮かんだ.いろんなセンサを取り付けて,複雑な意味合いの制御を強化学習で実現することを考えるのが良いのかもしれないと思った.\par
それと同時にMuJoCoを使うメリットよりもデメリットのほうが大きいと思い,2Dシミュレータを作ることを考えた.
2Dのほうが学習は高速にでき,シミュレータの内部で行われている計算も,作ることで間違いなく設定できるからである.\par
よってSim-to-Realを目指すが,研究の方向性は,今まで想定していた「Sim-to-Realの助けになる新たな工夫を発見すること」ではなく,「実機(2輪移動ロボット)のまだ誰もやっていない意味を持つ制御動作を,従来の強化学習でSim-to-Realすることで実現する研究」にしようかと考えている.\par
方向性はそうしようと思っているが,まだ具体的なアイデアはない.論文を探したりして考える必要がある.

\end{document}